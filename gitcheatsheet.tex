\documentclass[10pt,landscape]{article}
\usepackage{multicol}
\usepackage{calc}
\usepackage{ifthen}
\usepackage[landscape]{geometry}
\usepackage{amsmath,amsthm,amsfonts,amssymb}
\usepackage{color,graphicx,overpic}
\usepackage{hyperref}


\pdfinfo{
  /Title (example.pdf)
  /Creator (TeX)
  /Producer (pdfTeX 1.40.0)
  /Author (Seamus)
  /Subject (Example)
  /Keywords (pdflatex, latex,pdftex,tex)}

% This sets page margins to .5 inch if using letter paper, and to 1cm
% if using A4 paper. (This probably isn't strictly necessary.)
% If using another size paper, use default 1cm margins.
\ifthenelse{\lengthtest { \paperwidth = 11in}}
    { \geometry{top=.5in,left=.5in,right=.5in,bottom=.5in} }
    {\ifthenelse{ \lengthtest{ \paperwidth = 297mm}}
        {\geometry{top=1cm,left=1cm,right=1cm,bottom=1cm} }
        {\geometry{top=1cm,left=1cm,right=1cm,bottom=1cm} }
    }

% Turn off header and footer
\pagestyle{empty}

% Redefine section commands to use less space
\makeatletter
\renewcommand{\section}{\@startsection{section}{1}{0mm}%
                                {-1ex plus -.5ex minus -.2ex}%
                                {0.5ex plus .2ex}%x
                                {\normalfont\large\bfseries}}
\renewcommand{\subsection}{\@startsection{subsection}{2}{0mm}%
                                {-1explus -.5ex minus -.2ex}%
                                {0.5ex plus .2ex}%
                                {\normalfont\normalsize\bfseries}}
\renewcommand{\subsubsection}{\@startsection{subsubsection}{3}{0mm}%
                                {-1ex plus -.5ex minus -.2ex}%
                                {1ex plus .2ex}%
                                {\normalfont\small\bfseries}}
\makeatother

% Define BibTeX command
\def\BibTeX{{\rm B\kern-.05em{\sc i\kern-.025em b}\kern-.08em
    T\kern-.1667em\lower.7ex\hbox{E}\kern-.125emX}}

% Don't print section numbers
\setcounter{secnumdepth}{0}


\setlength{\parindent}{0pt}
\setlength{\parskip}{0pt plus 0.5ex}

%My Environments
\newtheorem{example}[section]{Example}
% -----------------------------------------------------------------------

\raggedcolumns

\begin{document}
\raggedright
%\raggedcolumns
\footnotesize
\begin{multicols}{3}
%\raggedcolumns


% multicol parameters
% These lengths are set only within the two main columns
%\setlength{\columnseprule}{0.25pt}
\setlength{\premulticols}{1pt}
\setlength{\postmulticols}{1pt}
\setlength{\multicolsep}{1pt}
\setlength{\columnsep}{2pt}

\begin{center}
     \Large{Git Cheatsheet}
	\vspace{2pt}
	\hrule
\end{center}

\section{Commands}

\begin{samepage}
\subsection{\texttt{Reset}}

\begin{verbatim}
git reset (--soft | --mixed | --hard) [-q] [<commit>]
\end{verbatim}

\begin{itemize}
	\item \texttt{--soft} - doesn't reset index nor working tree
	\item \texttt{--mixed} - reset index, not working tree
	\item \texttt{--hard} - reset index and working tree
\end{itemize}
\end{samepage}

\section{Tasks}

\subsection{Push Local Branch To Remote Repository}

\begin{verbatim}
git push -u repository-spec branch-name
\end{verbatim}

\subsection{Checkout Remote Branch}

\begin{verbatim}
git fetch repository
git checkout -b branch-name repository\/branch-name
\end{verbatim}

\begin{samepage}
\subsection{Reset (Revert) a Single File}

\begin{verbatim}
git checkout filename
\end{verbatim}

Or, in case there is a filename with the same name as one of the branches:

\begin{verbatim}
git checout -- filename
\end{verbatim}
\end{samepage}

% You can even have references
% \rule{0.3\linewidth}{0.25pt}
% \scriptsize
% \bibliographystyle{abstract}
% \bibliography{refFile}
\end{multicols}
\end{document}
